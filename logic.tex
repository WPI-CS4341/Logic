%% LyX 2.1.4 created this file.  For more info, see http://www.lyx.org/.
%% Do not edit unless you really know what you are doing.
\documentclass[english]{article}
\usepackage[T1]{fontenc}
\usepackage[latin9]{inputenc}
\usepackage{geometry}
\geometry{verbose,tmargin=1in,bmargin=1in,lmargin=1in,rmargin=1in}
\setlength{\parindent}{0bp}

\makeatletter

%%%%%%%%%%%%%%%%%%%%%%%%%%%%%% LyX specific LaTeX commands.
%% Because html converters don't know tabularnewline
\providecommand{\tabularnewline}{\\}

\makeatother

\usepackage{babel}
\begin{document}

\title{Logic Proofs}

\author{Harry Liu \& Tyler Nickerson}
\maketitle
\begin{enumerate}
\item \emph{Detective James is solving a case. The four suspects $A$, $B$,
$C$ and $D$ made the following statements:}

\begin{itemize}
\item \emph{If $A$ is guilty then $B$ was an accomplice}
\item \emph{If $B$ is guilty then either $C$ was an accomplice or A is
innocent}
\item \emph{If $D$ is guilty then $A$ is guilty and $C$ is innocent}
\item \emph{If $D$ is guilty then $A$ is guilty}
\end{itemize}

\emph{Is D guilty based on these statements? }


\subsubsection*{Proof By Truth Table}


First, let us convert the above statements to their CNF logical equivalents. To clean up our proof, let us assume that
if a constant $X$ is true, then it means that suspect $X$ is guilty. For this problem, the term ``accomplice'' will be used
interchangeablely with ``guilty''.

\begin{enumerate}
\item $\lnot A\lor B$
\item $\lnot B\lor(C\lor\lnot A)$
\item $(\lnot D\lor A)\land(\lnot D\lor\lnot C)$
\item $\lnot D\lor A$
\end{enumerate}

Next, we use these sentences to generate the following truth table:
\\\\
\begin{tabular}{ | l | l | l | l | l | l | l | l | }
\hline
	$A$ & $B$ & $C$  & $D$ & (a) & (b) & (c) & (d) \\ \hline
	T & T & T & F & T & T & T & T \\ \hline
	T & T & F & F & T & F & T & T \\ \hline
	T & F & T & F & F & T & T & T \\ \hline
	T & F & F & F & F & T & T & T \\ \hline
	F & T & T & F & T & T & T & T \\ \hline
	F & T & F & F & T & T & T & T \\ \hline
	F & F & T & F & T & T & T & T \\ \hline
	F & F & F & F & T & T & T & T \\ \hline
	T & T & F & T & T & F & T & T \\ \hline
	T & F & T & T & F & T & F & T \\ \hline
	T & F & F & T & F & T & T & T \\ \hline
	F & T & T & T & T & T & F & F \\ \hline
	F & T & F & T & T & T & F & F \\ \hline
	F & F & T & T & T & T & F & F \\ \hline
	F & F & F & T & T & T & F & F \\ \hline
	T & T & T & T & T & T & F & T \\ \hline
\end{tabular}
\\\\
Looking at this table, we see that for all cases in which $D$ is guilty,
there are no cases in which all four statements are true. Therefore, we cannot prove that $D$ is guilty.
\\\\
Following the above sentences, this makes sense. If $D$ is guilty,
then $A$ is guilty and $C$ is innocent. However, if $A$ is guilty, then $B$ was
an accomplice (guilty), and if $B$ is guilty, then $C$ was either an accomplice
(guilty) or $A$ was innocent. However, we just said $C$ is innocent and $A$ is
guilty! This is a contradiction, and we will prove this more formally later.

\subsubsection*{Proving By Resolution}

Let us formalize this proof using resolution:
\\\\
\begin{tabular}{lll}
1 & $\lnot A\lor B$ & Given \\
2 & $\lnot B\lor C\lor\lnot A$ & Given \\
3 & $\lnot D\lor A$ & Given \\
4 & $\lnot D\lor\lnot C$ & Given \\
5 & $B\lor\lnot D$ & Resolution between 1 and 3 \\
6 & $\lnot A \lor\lnot B \lor\lnot D$ & Resolution between 2 and 4 \\
7 & $\lnot A \lor C \lor\lnot D$ & Resolution between 2 and 5 \\
\end{tabular}
\\\\\\
No further resolutions can be drawn. As a result, we cannot prove $D$ is true/guilty, as it cannot be resolved
using the given statements.

\subsubsection*{Adding Refutation}

Now let us add $\lnot D$ to the premise:
\\\\
\begin{tabular}{lll}
1 & $\lnot A\lor B$ & Given \\
2 & $\lnot B\lor C\lor\lnot A$ & Given \\
3 & $\lnot D\lor A$ & Given \\
4 & $\lnot D\lor\lnot C$ & Given \\
5 & $\lnot D$ & Negated conclusion \\
6 & $B\lor\lnot D$ & Resolution between 1 and 3 \\
7 & $\lnot A \lor\lnot B \lor\lnot D$ & Resolution between 2 and 4 \\
8 & $\lnot A \lor C \lor\lnot D$ & Resolution between 2 and 5 \\
\end{tabular}
\\\\
We still cannot resolve any further, so we cannot prove whether suspect $D$ is guilty or not.

\item \emph{If the Congress refuses to vote for new laws then strike would
not be finished. Except for the case when it continues for more than
a month and a company's CEO retires. The Congress refuses to operate
and strike finishes. Therefore strike was going on for more than a
month.}

\subsubsection*{Proof by Truth Table}

Let us first convert these sentences into logical constants:

\begin{itemize}
    \item Let $A$ denote Congress refusing to vote for new laws
    \item Let $B$ denote the strike finishing
    \item Let $C$ denote the strike continuing for more than a month
    \item Let $D$ denote a company's CEO retiring
\end{itemize}

We can now incorporate these into CNF logical statements as so:

\begin{enumerate}
    \item $\lnot A \lor\lnot B$
    \item $(\lnot C \lor\lnot D) \lor B$
\end{enumerate}

which generates the following truth table:

\begin{tabular}{ | l | l | l | l | l | l | }
\hline
	$A$ & $B$ & $C$ & $D$ & $\lnot A \lor\lnot B$ & $(\lnot C \lor\lnot D) \lor\lnot B$ \\ \hline
    T & T & T & T & F & T \\ \hline
	T & T & T & F & F & T \\ \hline
	T & F & T & T & T & F \\ \hline
	T & F & T & F & T & T \\ \hline
	F & T & T & T & T & T \\ \hline
	F & T & T & F & T & T \\ \hline
	F & F & T & T & T & F \\ \hline
	F & F & T & F & T & T \\ \hline
	T & T & F & T & F & T \\ \hline
	T & T & F & F & F & T \\ \hline
	T & F & F & T & T & T \\ \hline
	T & F & F & F & T & T \\ \hline
	F & T & F & T & T & T \\ \hline
	F & T & F & F & T & T \\ \hline
	F & F & F & T & T & T \\ \hline
	F & F & F & F & T & T \\ \hline
\end{tabular}

Now, remember the statement we are trying to prove: Congress refused to operate, and the strike finished after lasting
more than a month. We check our knowledge base by looking at each instance in which $A$, $B$, and $C$ are true, then
checking to see if statements (a) and (b) are satisfied at these points.
\\
Examining the table, we

\subsubsection*{Proof by Resolution}

We formalize this proof through resolution:

\begin{tabular}{lll}
1 & $\lnot A \lor B$ & Given \\
2 & $\lnot B \lor C$ & Given \\
3 & $C$ & Given \\
4 & $\lnot A \lor C$ & Resolution of 1 and 2 \\
\end{tabular}


\item \emph{If 2 is a prime number then 2 is the smallest prime number.
If 2 is the smallest prime number then 1 is not a prime number. 1
is not a prime number. Are the following propositions correct based
on the aforementioned statements?}

\begin{itemize}
\item \emph{2 is the smallest prime number}
\item \emph{2 is a prime number }
\end{itemize}

\subsubsection*{Proof by Truth Table}

Let us convert these statements into logical constants. Let $A$ denote
2 being a prime number, $B$ denote 2 being the \emph{smallest} prime number,
and let $C$ denote 1 as not a prime number. Given these conversion, we
are left with the following sentences:

\begin{enumerate}
    \item $\lnot A \lor B$
    \item $\lnot B \lor C$
    \item $C$
\end{enumerate}

From this we create the following truth table:
\\\\
\begin{tabular}{ | l | l | l | l | l | l | }
\hline
	$A$ & $B$ & $C$ & $\lnot A \lor B$ & $\lnot B \lor C$ & $C$ = True \\ \hline
    T & T & T & T & T & T \\ \hline
	T & T & F & T & F & F \\ \hline
	T & F & T & F & T & T \\ \hline
	T & F & F & F & T & F \\ \hline
	F & T & T & T & T & T \\ \hline
	F & T & F & T & F & F \\ \hline
	F & F & T & T & T & T \\ \hline
	F & F & F & T & T & F \\ \hline
\end{tabular}
\\\\\\
Looking at this table, we can see that there are only two case in which all three of our
statements are true:

\begin{itemize}
    \item When $A$ and $B$ are both false and $C$ is true
    \item When $A$, $B$, and $C$ are all true
\end{itemize}

Therefore, there is only two solutions in which the above propositions are true. In other worlds, these propositions
cannot be proven.

\subsubsection*{Proof by Resolution}

We formalize this proof through resolution:

\begin{tabular}{lll}
1 & $\lnot A \lor B$ & Given \\
2 & $\lnot B \lor C$ & Given \\
3 & $C$ & Given \\
4 & $\lnot A \lor C$ & Resolution of 1 and 2 \\
\end{tabular}
\\\\\\
From this, we cannot resolve neither $A$ nor $B$. Therefore,
both cannot be proven.

\subsubsection*{Adding Refutation}

To further our proof, let us assume $B$ is false. We can add $\lnot B$
to the above table, however, we cannot resolve any further statements
using it. Therefore, we are left with the same knowledge base, and
we still cannot prove neither $A$ nor $B$.

\end{enumerate}
\end{document}
